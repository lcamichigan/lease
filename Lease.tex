\documentclass{article}

\input LeasePreamble

\begin{document}

\section*{Sigma Alumni Association \leaseDescription\ Lease}

This lease (the \definedTermDeclaration{Lease}) is made between
\emph{Sigma Alumni Association of Lambda Chi Alpha,~Inc.}, a Michigan nonprofit
corporation (the \definedTermDeclaration{Landlord}), and
\emph{\tenantName} (the \definedTermDeclaration{Tenant}). Tenant, whose address
is \emph{\tenantAddress}, is a member of \emph{Sigma Zeta} (the
\definedTermDeclaration{Chapter}) of \emph{Lambda Chi Alpha} (the
\definedTermDeclaration{Fraternity}). Landlord’s billing service (the
\definedTermDeclaration{Billing Service}) is \emph{GreekBill,~Inc.}, whose
address is \emph{23224 Crenshaw Blvd, Torrance CA 90505-3171}.

\subsection{Premises}
\emph{1601 Washtenaw Ave, Ann Arbor MI 48104-3586} is the
\definedTermDeclaration{Premises}. Landlord agrees to lease to Tenant a room in
the Premises. Tenant shall select this room according to the policies of the
Chapter, and Tenant shall have full, non-exclusive access to and use of this
room.

\subsection{Tenants and Common Areas}
Tenant together with other lessees of the Premises are the
\definedTermDeclaration{Tenants}. Tenant shall share non-exclusive use of common
areas of the Premises with other lessees of the Premises.

\subsection{Term}
The term of this Lease (the \definedTermDeclaration{Term}) is
\emph{\the\termDuration\ 24-hour periods} starting on
\emph{\DTMusedate{lease start date}} and ending on
\emph{\DTMusedate{lease end date}}.

\subsection{Rent}
All rent herein is reserved. Tenant shall pay to Landlord rent for the Term of
\emph{\rentTotal}, plus all other rent that becomes due under this Lease. Rent
shall be paid in monthly installments by the due dates in the first column of
the table below and in the amounts to the right of such dates:

{
  \addfontfeatures{Numbers=Monospaced}
  \hspace{-\tabcolsep}%
  \begin{tabular}{rl}
    \textbf{Rent Due Date} & \textbf{Rent}\\
    \rentTable
  \end{tabular}
}

Rent is paid only when received by Landlord. If Tenant’s rent check is returned
unpaid, Tenant shall pay as additional rent any returned check fees. The
foregoing fees shall be paid by the 15th~day of the calendar month following the
month in which they accrue. Landlord, in its sole discretion, may require
monthly installments to be paid with certified funds or money orders and in a
single payment.

Tenant shall not pay rent in cash to any person, including representatives of
Landlord or Billing Service.

\subsection{Late Lease Fee}
If Tenant delivers this signed Lease to Landlord after
\emph{\DTMusedate{lease due date}} (the
\definedTermDeclaration{Lease Due Date}), Tenant shall pay to Landlord as
additional rent a late lease fee of \emph{\$20.00} per month after the Lease Due
Date by the start of the Term of this Lease. For avoidance of doubt, if this
Lease is delivered from \DTMmonthname{\themonth}
\DTMsavedate{lease due date month end}{%
  \DTMfetchmonth{lease due date}-\DTMfetchmonth{lease due date}-last%
}%
2–\DTMfetchday{lease due date month end}, \DTMfetchyear{lease due date}, then
Tenant shall pay a late lease fee of \$20.00; if this Lease is delivered in
{%
  \DTMsetdatestyle{month year}%
  \stepcounter{month}%
  \DTMdate{\DTMfetchyear{lease due date}-\themonth-01}, then Tenant shall pay a
  late lease fee of \$40.00; if this Lease is delivered in
  \stepcounter{month}%
  \DTMdate{\DTMfetchyear{lease due date}-\themonth-01}, then Tenant shall pay a
  late lease fee of \$60.00; and so on.
}

\subsection{Security Deposit}
Tenant shall pay to Landlord a security deposit of \emph{\securityDeposit} by
\emph{\DTMusedate{security deposit due date}}. Landlord shall deposit Tenant’s
security deposit in \emph{\securityDepositLocation}.

If Tenant adheres to the provisions in this Lease, moves out of the Premises by
the end of the Term, and returns the Premises to Landlord in the same condition
it was in upon taking possession (other than normal wear and tear), Landlord
will return to Tenant the full amount of the security deposit within 30~days
after the end of the Term. If Tenant does not adhere to the provisions in this
Lease, Landlord may keep all or part of Tenant’s security deposit to pay
Landlord for any damages or losses incurred, including missed payments.

Landlord is not responsible for allocating shares of the security deposit or for
attributing individual liability for charges it makes against it. Damages to
Tenant’s room in the Premises shall be assessed to Tenant and/or other lessees
of the Premises at Landlord’s sole discretion. Damages to common areas in the
Premises shall be assessed to Tenant or Tenants at Landlord’s sole discretion.
Security Deposit Act communications shall be sent to Landlord. Tenant may not
elect to use the security deposit for rent. If Tenant causes damages that exceed
the security deposit, Tenant will pay those excess damages immediately upon
receiving notice thereof. If Landlord is in possession of a security deposit
from a prior lease period, Tenant consents that the security deposit may be held
by Landlord and applied to the security deposit required under this Lease, even
if the terms of the two leases are not consecutive or do not allow for
uninterrupted occupancy.

\subsection{Place of Payment and Notices}
Tenant’s rent shall be delivered or mailed to Billing Service. Landlord’s
notices to Tenant (other than security deposit claim notices) shall be delivered
or mailed to the Premises. Tenant’s notices to Landlord, including security
deposit communications, shall be delivered or mailed to \emph{Alumni Advisor
\AlumniAdvisorName, \AlumniAdvisorAddress}, or to another location if notice of
such location is provided to Tenant by Landlord in writing. Notices required by
this Lease or by law shall be in writing. Notices that are mailed (including
security deposit claim notices) are deemed received by the addressee on the next
regular day for delivery of mail after being properly addressed, stamped with
sufficient postage, and deposited in a United States mailbox.

\subsection{Cancellation Due to Associate Membership}
If Tenant is an associate member of the Fraternity on the Lease Due Date and
Tenant is not an initiated member of the Fraternity on
\DTMdate{\DTMfetchyear{lease due date}-05-last}, then either Tenant or Landlord
may end this Lease. If this Lease is so ended, any money paid by Tenant on
account of this Lease shall be refunded promptly by Landlord, and Tenant and
Landlord shall have no further obligations under this Lease.

\subsection{Application of Tenant’s Money}
Money received by Landlord from Tenant (or on Tenant’s behalf) shall be applied
to Tenant’s account as follows: first to satisfy late fees, returned check fees,
and to other fees owed by Tenant; second to maintenance and repair costs
chargeable to Tenant; third to collection fees, legal fees and court costs
legally chargeable to Tenant, including costs incurred prior to curing a
default; fourth to security or membership deposits or portions thereof due from
Tenant; fifth to charges, fines, and assessments against Landlord caused by
Tenant; sixth to rent; and seventh to food service charges, if any. Restrictive
language on a check or in any communication, including those accompanying a
payment, shall not constitute an accord and satisfaction or amend this
provision.

\subsection{Taking Possession and Moving In}
Tenant shall neither take possession of Tenant’s room in the Premises nor be
allowed to use the Premises’ common areas unless and until the late lease fee
(if any), first month’s rent, security deposit, and any other amounts due the
Landlord are paid. If Tenant does not take possession by the start of the Term
and Tenant has not notified Landlord in writing that Tenant will take possession
at a later time, Landlord may conclude that Tenant has abandoned the Premises
and re-rent them. If the Premises are not ready for occupancy at the start of
the Term, Landlord’s sole liability to Tenant is abatement of Tenant’s rent, in
the same percentage that the Premises are not ready for occupancy, from the
start of the Term to when the Premises are ready for occupancy, which time is at
Landlord’s exclusive determination.

Tenant has received a Move-In Form and shall return a completed copy of it to
Landlord within 7~days after moving in. The Premises are conclusively presumed
to be in good condition upon moving in unless Tenant specifies objections on the
Move-In Form. Such objections are not a request for repairs.

\subsection{Maintenance}
Tenant shall maintain the Premises in a neat, clean, and orderly manner; use and
maintain them in accordance with applicable police, sanitary, and all other
regulations imposed by governmental authorities; observe all reasonable
regulations and requirements of underwriters concerning use and condition of the
Premises tending to reduce fire hazard and insurance rates; and immediately
inform Landlord when there is a need for Landlord to perform repairs or
maintenance. Tenant shall not cause or permit any waste or misuse of any utility
fixtures or of any portion of the Premises. During the Term of this Lease,
Tenant is responsible for all damages to the Premises caused by Tenant, Tenant’s
guests, invitees, licensees, agents or by unknown parties. Tenant shall
reimburse Landlord for all damages caused by such waste or misuse; for all
permits, inspection, and certification costs Landlord incurs because of Tenant’s
noncompliance with this Lease or applicable laws; and for all damages resulting
from Tenant’s not timely reporting the need for repair or maintenance. Landlord
may invoice Tenant for the cost of any repairs\slash replacements (other than
normal wear and tear) during the Term of this Lease. The amount of such invoices
is deemed unpaid rent and shall be due with the rent installment for the month
following the month in which the invoice is sent. Unless authorized Chapter
representatives designate to which Tenant invoices for repairs or damages should
be allocated, invoices shall be allocated equally among all Tenants. Tenant
shall pay and be liable to Landlord and/or Landlord’s insurer (in contract
and/or tort) for the repair of all damage to, and/or replacement of, the
Premises and structure of which they are a part, including fire and flood
damage, in any way caused or made necessary by Tenant, Tenant’s guests,
invitees, licensees, or agents. Nothing in this provision waives or lessens
Landlord’s obligation to maintain and repair the Premises under Michigan law,
but Landlord is not so liable when it has not been informed of the need to
repair. Landlord’s reasonable exercise of any right or obligation hereunder
never shall be deemed an eviction of Tenant or interference with Tenant’s use
and possession of the Premises, and Landlord shall have no liability to Tenant
because of Landlord’s actions in reasonably fulfilling its obligations
hereunder.

\subsection{Moving Out}
Tenant shall return the Premises at the expiration of the Term (or earlier
termination) in as good a condition as when received, reasonable wear and tear
excepted. Early surrender of the Premises, including surrender accepted in
writing, does not extinguish any of Tenant’s obligations to perform under this
Lease, including payment of all rent required herein.

\subsection{Alterations}
Tenant shall not alter or remodel the Premises (or any room or portion thereof)
without prior written consent of Landlord. Tenant may, within the bounds of
Tenant’s room, install flooring, install non-permanent decorations, or construct
self-supporting sleeping lofts in a manner so as to not damage the ceiling,
walls or floors and in compliance with all building, fire and safety codes.
Tenant may paint the walls and ceilings a neutral white or beige color. If
Tenant paints the walls or ceilings any other color, this will be considered
damage and Tenant may be charged the cost to repaint such areas. If Tenant
desires to determine if a specific color is acceptable, Tenant should obtain
advance, written permission from Landlord.

Landlord is not responsible for the floor coverings, window coverings, wall and
ceiling colors, or decorating in a sleeping room. Tenant will remove, repair and
return each room to the same condition as delivered, including the removal of
all non-permanent decorating including tape, adhesives, nails, poster putty,
\textit{etc}. Floor coverings, wall colors and ceiling colors do not have to be
restored and may remain at the end of the Term. Tenant agrees to pay all costs
incurred by Landlord to remove or repair any prohibited alteration and to
restore the Premises to its original condition.

\subsection{Utility Charges}
Utility charges relating to the Premises, including water, sewer, gas,
electricity, and alarm monitoring services shall be paid by Landlord. Landlord
is not responsible for utility service interruptions that are beyond its control
or due to necessary repairs, replacements, or alterations.

\subsection{Internet and Television}
Landlord shall arrange and pay for internet service including any necessary
routers. Landlord shall arrange and pay for basic cable television services.
Tenant shall arrange and pay for any necessary cable television boxes or
decoders. Tenant may not modify internet or cable television services. Tenant
may not install any new wiring or make any alterations to the Premises without
the prior written approval of Landlord. Any unauthorized changes or alterations
shall be considered damages and Tenant shall be responsible for the costs to
fully restore any alterations.

\subsection{Entry by Landlord}
Landlord may enter the Premises in an emergency or to perform repairs,
maintenance, inspections, appraisals, insurance inspections, other purposes
reasonably related to the operation of the building, and to show the Premises
for sale or lease. Except during an emergency or when Landlord reasonably
believes there is an emergency, all entries shall be made during reasonable
hours, and Landlord shall make reasonable efforts to inform Tenant of its
intention to enter Tenant’s room and to establish a mutually acceptable time.
Tenant and Landlord agree that access to the common areas of the Premises during
reasonable hours for the purposes stated herein do not require notice to
Tenant.

\subsection{Default and Remedies}\label{Default and Remedies}
Tenant’s noncompliance with any provision of this Lease is a default. Tenant
shall also be in default and Landlord may (but is not obligated to) terminate
this Lease by written notice, as required by law, upon occurrence of any of the
following events:

\begin{compactList}{(\alph{enumi})}{\usecounter{enumi}}
  \item Tenant pays rent late on three or more occasions.
  \item The Fraternity revokes the Chapter’s charter.
  \item Tenant is expelled or resigns from the Chapter or the Fraternity.
  \item Tenant is expelled from the University of Michigan.
  \item Tenant is convicted of or pleads no contest to a felony crime.
\end{compactList}

If Tenant defaults, Landlord may have all remedies legally permitted, including
termination of this Lease by written notice and declaring all remaining rental
installments immediately due and owing. Landlord may terminate this Lease on
24~hours written notice if Tenant, a member of Tenant’s family, or a guest of
Tenant unlawfully manufactures, delivers, possesses with intent to deliver, or
possesses a controlled substance on the Premises. Tenant shall reimburse
Landlord for all legal fees, costs, and expenses legally recoverable and for all
damages caused by Tenant’s default, including costs of re-renting the Premises,
such as showing, advertising, and preparing the Premises; unpaid rent up to the
end of this Lease; and the maximum amount of interest allowed by Michigan law on
Tenant’s debt, from the date Tenant vacates. Tenants may not be liable for the
total accelerated amount because of Landlord’s obligation to minimize its
damages, and either Tenant or Landlord may have a court determine the actual
amount owed. Once this Lease is ended, whether because of default or otherwise,
Tenant waives any and all rights to reinstate or renew this Lease.

\subsection{Mediation}
Landlord and/or Tenant may request that the University of Michigan Off-Campus
Housing Program assist in disputes, \provisoMarker{provided} that Tenant is a
student of the University of Michigan. Landlord and Tenant shall make a
reasonable and good faith effort to settle such disputes through the Off-Campus
Housing Program. Off-Campus Housing Program staff may enter and inspect the
Premises after notice to Landlord and Tenant and at reasonable times. This
provision shall not preclude other legal rights of Landlord or Tenant. Landlord
and Tenant agree to keep mediation proceedings confidential.

\subsection{Hold Harmless}
Tenant agrees for himself, his heirs, and personal representatives to hold
Landlord harmless from all damages, including damages to the Premises and
structure of which they are a part; all lost rents for the Premises and
structure of which they are a part; and all liability that results from his
negligent or illegal use of the Premises and from his intentional misuse of
them, including common areas. When claims against Landlord’s insurance are paid
because of acts or omissions of Tenant, Tenant’s visitors, guests, or invitees,
Tenant will reimburse Landlord for any insurance deductible it pays.

\subsection{Damage to Tenant’s Property}
Landlord does not provide any insurance coverage for personal property of
Tenant, Tenant’s guests or invitees, and shall not be liable for any damage,
loss, or destruction of such property from any cause, including acts or
omissions of third parties, unless caused by Landlord’s nonperformance or
negligent performance of a duty imposed by law or by their grossly negligent or
intentional actions. Tenant specifically is encouraged to insure Tenant’s
personal property.

\subsection{Amendment}
This Lease may be amended in writing only, signed by Landlord and Tenant; except
that on 30~days written notice to Tenant, Landlord may (i)~amend this Lease to
conform with changes required by federal, state, or local law, rule, or
regulation or to implement changes in rules relating to the Premises which are
required to protect the physical health, safety, or peaceful enjoyment of
tenants and guests; and (ii)~increase required rental payments to cover
additional costs incurred by Landlord in operating the Premises because of any
increase in \textit{ad valorem} property taxes, charges for the electricity,
heating fuel, water consumed at the property, or increases in premiums paid for
liability, fire, or worker compensation insurance. Upon giving such notice, this
Lease is deemed amended in accord with the terms specified in the notice. If the
new rental rate starts other than on the 15th~day of a calendar month, the
amount of increase due from the start date to the 15th~day of the next calendar
month will be pro-rated on a 30-day month basis and due as part of the rent
installment for that next calendar month.

\subsection{Waiver}
Landlord’s non-enforcement of a provision of this Lease on one or more occasions
is not a continuing waiver of Landlord’s right to enforce the provision, and its
consent to an act of Tenant on one or more occasions (where consent is required)
is not a continuing consent to any subsequent similar act by Tenant. No breach
is waived by Landlord unless waived in writing.

\subsection{Severability}
Neither a court ruling that a portion of this Lease is invalid nor Landlord’s
and Tenant’s written agreement not to observe a portion of this Lease shall
invalidate any other portions of this Lease.

\subsection{Successors Bound}
Heirs, successors, assigns, and representatives of Landlord and Tenant shall be
bound by the provisions of this Lease.

\subsection{Use and Quiet Enjoyment}
Tenant shall comply with all applicable laws and ordinances; use the Premises
only for strictly residential purposes; and refrain from all conduct that
unreasonably disturbs other tenants, occupants, neighbors of the building, or
Landlord. No business of any sort shall be located in or conducted from the
Premises. Tenant is entitled to quiet enjoyment of the Premises throughout the
Term of this Lease so long as Tenant complies with the Lease’s provisions.

\subsection{Untenantability}
If the Premises become wholly untenantable because of fire or other casualty,
Landlord may terminate this Lease by written notice to Tenant, and Tenant shall
surrender the Premises to Landlord. If for the same reasons the Premises become
partially untenantable, or wholly untenantable without Landlord’s terminating
this Lease, Landlord shall repair the Premises with reasonable speed. From the
date of the casualty, until repairs are substantially completed, rent shall
abate in the same percentage that the Premises are untenantable, unless the
untenantability is caused by negligence or intentional misconduct of Tenant, his
family, occupants, employees, guests, invitees, agents, or anyone on the Premises
by reason of association with any of them, in which case rent shall not abate.
Landlord is not liable for failure to repair until Tenant notifies Landlord of
the need for repair and a reasonable time to make the repair has passed
thereafter. For purposes of Landlord’s right to terminate this Lease, the
Premises are \definedTermDeclaration{wholly untenantable} if 50\% or more of the
Premises are untenantable.

\subsection{Assignment, Subletting, and Occupancy}
Tenant may not assign this Lease or sublet any of the Premises without
Landlord’s prior written permission. Landlord may require that any lease
assignment or sublet is to another member of the Chapter. Landlord may evaluate
proposed assignees and subtenants as it evaluates prospective Tenant, including
acceptability to remaining prime Tenant. Only Tenant may occupy the Premises.
Tenant may not have any guest occupy Tenant’s room for more than 4~consecutive
days.

\subsection{Abandonment}
During the Term, if Landlord believes that Tenant has abandoned the Premises and
current rent is unpaid, Landlord may enter the Premises and remove remaining
possessions of Tenant without liability therefor. Abandonment is presumed
conclusively if rent is unpaid for 15~days following the due date and (i)~a
substantial portion of Tenant’s possessions have been removed or
(ii)~acquaintances of Tenant or other reliable sources advise Landlord that
Tenant has left without intending to re-occupy the Premises. If Tenant abandons
or surrenders the Premises at any time and leaves personal property there,
Landlord may dispose of it however Landlord chooses, and Tenant shall reimburse
Landlord for all costs it incurs in that regard.

\subsection{Holding Over}
Tenant shall vacate the Premises on or before the end of the Term. If Tenant
retains possession thereafter without Landlord’s written permission, Landlord
has 30~days from the end of the Term to sue for possession under section
600.5714(1)(C)(2) of the Michigan Summary Proceedings Act (Holding over after
lease expires). If suit is not begun within that time, the tenancy shall
continue on a month-to-month basis from the end of the Term, and all other
provisions of this Lease shall remain in full force and effect; except that rent
shall increase by 50\%, beginning on the first day after the Term, regardless of
whether suit is brought or the tenancy becomes month-to-month. Landlord’s
acceptance of money from Tenant during the 30 days following the end of the Term
does not waive Landlord’s right to seek possession as described in this
provision, and Tenant shall compensate Landlord for all damages caused by
Tenant’s unauthorized holdover.

\subsection{Entire Agreement}\label{Entire Agreement}
This Lease, including the Rules and Regulations Addendum, is the entire
agreement of Landlord and Tenant, and they enter it voluntarily. There are no
other agreements that are part of this Lease or to which Landlord and Tenant are
bound unless specifically enumerated herein.

\subsection{Property Insurance}
Landlord shall, in its sole discretion, obtain property insurance on the
Premises and owner’s liability insurance. Landlord shall pay all such property
and liability insurance premiums and real estate taxes.

\subsection{Waiver of Subrogation}
Each policy of insurance authorized or required of either party under this Lease
shall contain a clause or endorsement under which the insurer waives all right
of subrogation against the other party, its agents and employees with respect to
losses payable under such policy, and each party hereby waives all right of
recovery it might otherwise have against the other party, its agents and
employees for any loss or injury which is covered by such a policy of insurance,
notwithstanding that such loss or injury may result from the negligence or fault
of such other party, its agents or employees.

\subsection{Time}
Time is of the essence of this Lease.

\subsection{Captions}
Captions are to assist with identification and have no legal significance.

% The section before the signature page is for notices required by the laws of
% Michigan and Ann Arbor. U-M’s off-campus housing webpage
% <https://offcampus.umich.edu/article/mandatory-lease-clauses> is a good
% resource for these notices.
\subsection{Mandatory Notices}
\begin{compactList}{(\alph{enumi})}{\usecounter{enumi}}
  % This notice is required by the Michigan Compiled Laws §554.634
  % <http://www.legislature.mi.gov/(S(s4mdwakzzlxktghbmm0v04n3))/mileg.aspx?page=getobject&objectname=mcl-554-634>.
  \item {\titleStyle Michigan Truth in Renting Act Notice.}\par
  {\fontsize{14}{17}\selectfont\bfseries%
    NOTICE: Michigan law establishes rights and obligations for parties to
    rental agreements. This agreement is required to comply with the Truth in
    Renting Act. If you have a question about the interpretation or legality of
    a provision of this agreement, you may want to seek assistance from a lawyer
    or other qualified person.\par
  }

  % This notice is required by Ann Arbor Housing Code §8:529(2)(c)
  % <https://www.municode.com/library/mi/ann_arbor/codes/code_of_ordinances?nodeId=TITVIIIBURE_CH105HOCO_8_529PR>.
  \item {\titleStyle City of Ann Arbor Privacy Ordinance.}\par
  {\fontsize{14}{17}\selectfont\bfseries\sloppy\MakeUppercase{%
    Notice: You have the right to privacy in your rental home. City law
    establishes guidelines that the owner and her\slash his agents must follow
    before entering your home. You may initiate additional entry restrictions by
    giving written notice to your landlord. Copies of these guidelines (Housing
    Code 8:529) are available at the Building Department, City Hall, 100
    N.~Fifth Ave.}\par\fussy
  }
  The address of Ann Arbor City Hall in this notice is as given in Housing Code
  8:529(2)(c), but Ann Arbor City Hall is located at 301 E~Huron~St.

  % The Michigan Compiled Laws §554.603, aka Act 348 of 1972
  % <http://www.legislature.mi.gov/(S(gtlfdjo0au0tsvyhnp4xxvoy))/mileg.aspx?page=getobject&objectname=mcl-554-603>,
  % requires a particular notice about security deposits. The act says that the
  % notice shall be “in 12 point boldface type which is at least 4 points larger
  % than the body of the notice or lease agreement” … so use 14 pt to be safe.
  % Ann Arbor City Charter §19.4(g) provides the text of a particular notice
  % that must be in leases, and §19.4(b) seems to say that this notice must
  % appear below the notice required by “Act 348, 1972”.
  \clearpage
  \item {\titleStyle Michigan Landlord-Tenant Act and City of Ann Arbor Truth in
  Renting Act Notices.}

  {\fontsize{14}{17}\selectfont\bfseries
    You must notify your landlord in writing within 4~days after you move of a
    forwarding address where you can be reached and where you will receive mail;
    otherwise your landlord shall be relieved of sending you an itemized list of
    damages and the penalties adherent to that failure.\par
  }
  % Among other things, Ann Arbor City Charter §19.4(f) requires “a solid black
  % border at least one-eighth inch thick”, …
  \fboxrule0.125in
  % … “at least one-half inch white space between the interior of the border and
  % the notice type”, …
  \fboxsep0.5in
  % and “at least one inch white space between the exterior of the border and
  % any type outside of the border”.
  \vspace{1in}
  \fbox{%
    \parbox{\dimexpr\linewidth-2\fboxsep-2\fboxrule\relax}{\parindent0.25in%
      % Ann Arbor City Charter §19.4(e) says, “The notice shall be in eleven
      % point boldface type or larger.”
      \fontsize{12}{14.5}\selectfont\bfseries%
      Some things your landlord writes in the lease or says to you may not be
      correct representations of your rights.

      Also, you may have rights and duties not mentioned in your lease. Such
      rights may include rights to repairs, rights to withhold rent to get
      repairs done, and rights to join a tenants union or to form your own
      union. Such duties may include the duty to pay rent due and the duty not
      to cause a serious health hazard or damage beyond reasonable wear and
      tear.

      Additionally, some lease clauses may be subject to differing legal
      interpretations. If you think that a clause in your lease or something
      your landlord says to you is unfair, you may contact your own lawyer,
      legal aid society, or tenants union lawyer for their opinions.
    }%
  }%
\end{compactList}

\clearpage % Start the signature page.

\MakeUppercase{In witness whereof}, Sigma Alumni Association of Lambda Chi
Alpha, Inc. and \tenantName\ have signed this Lease.

\begin{signatureBlock}
  Sigma Alumni Association of Lambda Chi Alpha, Inc.

  \signatureLine{\signature{support/SigmaSignature}\hfill\SigmaSignatureDate}

  By: \SigmaSignatoryName\hfill Date\\
  Its: \SigmaSignatoryTitle
\end{signatureBlock}

\begin{signatureBlock}
  Tenant

  \signatureLine{}

  \tenantName\hfill Date
\end{signatureBlock}

\vfill

\emph{\guarantorName} (the \definedTermDeclaration{Guarantor}), whose address is
\emph{\guarantorAddress}, agrees that if Tenant, for whatever reason, fails to
pay on time and in full any monetary amount due to Landlord under this Lease,
then Guarantor will promptly pay such amount in full upon notice by Landlord
that such amount is due and payable from Tenant. Guarantor waives notice of the
acceptance of this guarantee. Guarantor further waives presentment, protest, or
notice of any nature with respect to any monetary amounts due from Tenant under
this Lease, provided that Guarantor does not waive rights to service of process
or any defenses available to Tenant. This guarantee shall be binding upon
Guarantor, and upon heirs and executors thereof, and shall inure to the benefit
of Landlord and its respective successors and assigns. The foregoing guarantee
is not intended to create, with respect to Guarantor, any obligations of tenancy
or any liability to third parties.

\begin{signatureBlock}
  Guarantor

  \signatureLine{}

  \guarantorName\hfill Date
\end{signatureBlock}

\clearpage % Start the Rules and Regulations Addendum on a new page.

\section*{Rules and Regulations Addendum}
\setcounter{subsection}{0}

The following rules and regulations are part of this Lease as provided by
Section~\ref{Entire Agreement} on page~\pageref{Entire Agreement}.

\subsection{Furnishings}
The Premises are partially furnished. All furnishings in the Premises are part
of the Premises. Landlord agrees to furnish Tenant with one bed, one desk, one
desk chair and one built-in clothes closet and drawers for Tenant’s use.
Landlord has provided to Tenant certain common areas and common furnishings,
fixtures, and equipment that Tenant is entitled to use during the Term. Tenant
shall maintain said common areas and common furnishings, fixtures, and equipment
in clean and good condition. Tenant shall not remove any such furnishings,
fixtures, appliances or equipment. If any furnishings remain from prior Tenants,
Tenant may utilize those furnishings.

\subsection{Keys}
Landlord may retain a key to the Premises throughout the Term. Tenant shall not
change any lock without Landlord’s prior written consent. Landlord may also
charge Tenant a reasonable amount for assisting Tenant in gaining entry to the
Premises. Upon termination of this Lease, all keys shall be labeled and returned
o the Chapter president or house manager, and the Chapter’s representative shall
return all keys to Landlord when the Premises have been vacated. If the key to
Tenant’s room is not returned to Landlord, Tenant shall pay as additional rent
\$25.00. Keys returned to Landlord individually or after the date of vacating
the Premises shall not waive such charges.

\subsection{Parking}
Tenant may be assigned one or more vehicle parking spaces available at the
Premises according to the policies of the Chapter. Landlord makes no warranties
or guarantees that vehicle parking spaces sufficient to accommodate the vehicles
of all Tenants will be available at the Premises. Tenant shall park only on the
paved parking lot. Landlord may designate parking procedures for authorized
vehicles and have unauthorized vehicles, including vehicles that are disabled or
without current license plates, removed without notice and at Tenants’ expense.
Tenant shall not park commercial vehicles, boats, motorhomes, or trailers on the
Premises.

\subsection{Alarm Charges}
For each false alarm triggered at the Premises due to activities of Tenant or
Tenant’s guests that results in the dispatch of fire department personnel,
Landlord, in his sole discretion, may (i)~increase the amount of Tenant’s next
rent installment by the amount of all costs, fines and/or penalties incurred by
Landlord, or (ii)~increase the next rent installment of all Tenants by the
foregoing amount divided by the number of Tenants.

\subsection{Tenant Damages}
Landlord shall keep in good repair the roof, exterior walls, interior
load-bearing walls and alarm, heating, water, sewer, plumbing, and electrical
systems. Tenant shall protect all such fixtures and systems against damage.

\subsection{Care of Premises}
\begin{compactList}{\Alph{enumi}.}{\usecounter{enumi}}
  \item {\titleStyle Housekeeping.} Tenant shall perform regular cleaning and
  housekeeping of the rooms, kitchen, basement, bathrooms and other common areas
  of the Premises. If Landlord inspects any of the common areas and determines,
  in his sole discretion, that such areas require cleaning, Tenant shall be
  given notice therefor, and Tenant shall have 24~hours to restore the common
  areas to a condition acceptable to Landlord. If the common areas are not
  restored to a condition acceptable to Landlord after 24~hours, Landlord may
  cause such cleaning to be performed, and Tenant shall pay as additional rent
  all or a portion (decided by Landlord) of any cleaning costs incurred by
  Landlord.

  \item {\titleStyle Waste Removal.} If the Premises are cited for excessive
  garbage or debris on the property by Ann Arbor City officials, Tenants shall
  be responsible for the costs of any fines or other costs incurred.

  \item {\titleStyle Extermination.} Landlord shall be responsible, at its
  expense, for labor, materials, equipment and costs of insect and pest removal
  and extermination services. Tenant shall not allow conditions to exist that
  would invite insects and pests on the Premises. Tenant agrees to assist in
  such matters by placing mouse traps, fly strips, \textit{etc.} for minor
  infestations.

  \item {\titleStyle Supplies and Kitchenware.} Tenants shall be responsible for
  providing cleaning supplies, trash bags, trash containers, and cleaning
  equipment. Tenants shall provide their own paper products, such as toilet
  paper, paper towels, \textit{etc}. Tenants shall be responsible for furnishing
  and maintaining all kitchenware including dishes, glasses, cups, pots, pans,
  silverware, serving dishes, \textit{etc}. Tenants shall replace disposable
  items that may stop working during the Term of this Lease such as light bulbs,
  smoke detector batteries, \textit{etc}. If Landlord must replace disposable
  items that are Tenants’ responsibilities, Tenants will be responsible for the
  costs of replacement, including reasonable labor costs.

  \item {\titleStyle Lawn and Landscaping.} Tenants are responsible, at their
  expense for labor, materials, equipment and costs related to maintaining the
  lawn and landscaping.
\end{compactList}

\subsection{Use of Premises}
Use of the Premises shall be limited to Tenants, Chapter members, Chapter alumni
and their respective guests. Landlord reserves the right to host social events,
meetings and gatherings in the common areas of the Premises, in a manner that
does not interfere with the rights of Tenant. Landlord shall give Tenant
reasonable advance notice of such events.

\subsection{Rules and Policies}
For each violation of the following rules and policies, Landlord may assess
Tenant a fine of \$100.00 plus any costs incurred and/or terminate Tenant’s
right to occupy the Premises as provided in Section~\ref{Default and Remedies}
on page~\pageref{Default and Remedies} of this Lease.

\begin{compactList}{\Alph{enumi}.}{\usecounter{enumi}}
  \item {\titleStyle Laws and Controlled Substances.} Tenant shall abide by all
  federal, state, city, and county laws, ordinances and regulations. Tenant
  releases Landlord from any and all claims for personal injury to himself, his
  guests, and his related interests caused by the violation of law or the use of
  alcohol and/or controlled substances.

  \item {\titleStyle Noise and Social Gatherings.} Tenant shall abide by all
  rules and regulations related to noise and social gatherings imposed by the
  City of Ann Arbor. If Chapter is issued noise violations or other citations by
  the police or city authorities, Landlord shall not be responsible for noise
  violations or other citations imposed on occupants of the Premises.

  \item {\titleStyle Pets.} Tenant shall neither keep nor bring pets onto the
  Premises without the prior written consent of Landlord. (Service dogs of
  Tenant or Tenant’s guests that meet the identification and training
  verification requirements of Michigan law are not pets.) Tenants shall be
  collectively charged \$100.00 per day for any period in which they maintain a
  pet on the Premises without the prior written approval of the Landlord.
  Additionally, Tenants shall be responsible for any repairs, fumigations,
  cleaning or flooring replacement related to maintaining unauthorized pets on
  the Premises.

  \item {\titleStyle Fraternity and University Policies.} Tenant shall abide by
  all policies, rules, and regulations relating to new member recruitment,
  alcohol, controlled substances, and hazing imposed by the Fraternity, the
  Chapter, the University of Michigan, and the University of Michigan
  Interfraternity Council, which are incorporated by reference as if set forth
  herein in full.

  \item {\titleStyle Prohibited Items.} Tenant and Tenant’s guests shall not
  possess, store, or use on the Premises any item that is explosive, incendiary,
  or capable of causing significant bodily harm when used as intended, including
  but not limited to firearms, bows, arrows, crossbows, explosives, fireworks,
  and flammable items. For avoidance of doubt, Tenant and Tenant’s guests shall
  not possess firearms on the Premises even if Tenant and Tenant’s guests all
  have valid Concealed Pistol Licenses. Tenant shall not burn candles, incense,
  or other substances except during and as prescribed in official Fraternity
  ceremonies. Tenant shall not possess a fuel-burning space heater on the
  Premises.

  \item {\titleStyle Microwaves.} Tenant may possess and use 1 microwave in
  Tenant’s room in the Premises. Tenant shall not possess any additional cooking
  appliances except for those furnished by Landlord on the Premises.

  \item {\titleStyle Refrigerators.} Tenant may possess and use 1 miniature,
  compact refrigerator in Tenant’s room in the Premises.

  \item {\titleStyle Smoking.} Tenant shall not smoke any substance within the
  Premises.

  \item {\titleStyle Roof Access.} Tenant shall not enter onto any roof or into
  any attic of the Premises. Tenant shall not provide access to a roof or attic
  other than to Landlord or its designated contractors and agents.

  \item {\titleStyle Water-Filled Furniture.} Tenant shall not bring any
  water-filled furniture anywhere on the Premises.

  \item {\titleStyle Water, Sand, and Decorations.} Tenant shall not use
  liquids, sand, or other items that may cause damage to property within the
  Premises for the purposes of decoration. Tenant shall not construct pools,
  foam systems, slides, waterfalls, or water containment systems at or in the
  Premises.

  \item {\titleStyle Storage in Common Areas.} Tenant shall not store bicycles
  in the Premises anywhere but the first floor patio room. Tenant shall not
  store anything else in any of the Premises’ common areas, including but not
  limited to the basement, attic, hallways, laundry room, balconies, patios,
  porches, and fire escapes. Improperly stored items shall be deemed abandoned
  and disposed of without compensation or notice to Tenant.

  \item {\titleStyle Balconies, Porches, and Patios.} Tenant shall not place
  furniture on balconies, patios, or porches of the Premises unless such
  furniture is designed to be used outdoors. Cooking with a charcoal or gas
  broiler or any other device shall not be allowed at any time on balconies,
  porches, fire escapes, or under eaves or overhangs, and such cooking devices
  may only be used if they are at least 10 feet away from such structures.

  \item {\titleStyle Electricity.} Tenant shall use no more equipment than the
  Premises’ electrical system can safely handle. Tenant shall reset circuit
  breakers if necessary. Landlord does not warrant that the electrical system is
  compatible with computers or other sensitive electronic equipment and accepts
  no liability for damage to Tenant’s personal property caused by use of the
  electrical system unless such damage is the result of negligence by Landlord.
  Tenant shall not add any additional appliances (such as electric heaters, air
  conditioners, stoves, full-size refrigerators, freezers, washers, dryers,
  \textit{etc.}) without the written permission of Landlord.
\end{compactList}

\subsection{Tenant Liability Insurance}
Tenant shall participate in and pay the individual premium for the Fraternity’s
liability insurance program. Tenant hereby releases and agrees to indemnify
Landlord from and against any and all claims for personal injury as well as
claims for loss, destruction, or theft of personal property sustained by or
brought by Tenant or guests of Tenant while in or on the Premises, unless caused
by Landlord’s nonperformance or negligent performance of a duty imposed by law
or by their grossly negligent or intentional actions.

% Put a “Disclosure of Information on Lead-Based Paint”
% <https://www.epa.gov/sites/production/files/documents/lesr_eng.pdf> on its own
% page. Use \newgeometry from the geometry package to create a new page with a
% top margin of 0.
\newgeometry{top=0sp}
% Setting the inner (left) margin to 0 causes an issue with footer placement.
% As a workaround, add negative horizontal space to eliminate the inner margin
% <https://en.wikibooks.org/wiki/LaTeX/Page_Layout#Page_dimensions>.
\hspace*{\dimexpr-\oddsidemargin-1in-\hoffset\relax}%
% Draw the form using TikZ/PGF <https://www.ctan.org/pkg/pgf> so that check
% marks, a signature, and a date can be included.
\begin{tikzpicture}[
    % Prevent a page break before the picture.
    overlay,
    % Remove padding and margins around all nodes.
    every node/.style={inner sep=0,outer sep=0}
  ]
  \node at (106bp, -284bp) {✔};
  \node at (106bp, -415bp) {✔};
  \node[anchor=north west] {%
    % The top margin of the form is a little too large, so trim a bit off.
    \includegraphics[trim={0 0 0 12bp},clip]{support/lesr_eng}%
  };
  \node[anchor=south west] at (60bp, -656bp) {%
    \signature{support/SigmaSignature}%
  };
  \node[anchor=south east] at (303bp, -658.5bp) {\SigmaSignatureDate};
\end{tikzpicture}
\restoregeometry

\end{document}
